\section{VASM}

VASM is the official assembler shipped together with \emph{emulathor}. 

\section{Invoking VASM}

VASM requires at least an input file and an output file. To assemble a file the following command line is usually enough:
\verb|./vasm -v input.V -i output.img|. To create an image file with a fixed size an additional parameter may be given:
\verb|./vasm -v input.V -i output.img -f 1024| (create an image file with 1KB in size). VASM will abort with an error if the image file
grows over the specified fixed size. 

\section{Images}

VASM produces binary image files that can be directly loaded by an emulator.
These images files consist of raw binary instruction data. 

\section{Syntax}

The syntax is case-insensitive. 

\subsection{Comments}

The start of a comment is indicated by a semicolon \verb|;|. A comment is automatically terminated by a newline (and can not be terminated else). 

\subsection{OpCodes}

VASM uses the official OpCode Mnemonic as used in the official VA Documentation. 

\subsection{Operands}

\subsubsection{Registers}

VASM also uses the official Register Mnemonic as used in the official VA Documentation. VASM will detect usage of 8 bit or 16 bit registers
and automatically assume the corresponding Instruction Prefix to specify the operand size. If the instruction does not support the given operand size
VASM will abort with an error.

\begin{verbatim}
loadid r0, r1, 4
loadid r0l, r1, 4
loadid r0ll, r1, 4
\end{verbatim}

\subsubsection{Immediate Data}

\elparagraph{32 bit}

32 bit Immediate Data may either be specified in Hex Format or Integer Format. For Integer Format 32 bit Immediate Data requires an \verb|@| prefix.
For Hex Format this prefix is only required if not exactly eight hexadecimal digits are given. If Hex Format is used the Immediate Data is treated
unsigned else it is treated as a signed integer. 

\begin{verbatim}
load32 r0, @1232
load32 r1, 0x12345678
load32 r2, @0x123
\end{verbatim}

\elparagraph{16 bit}

16 bit Immediate Data may either be specified in Hex Format or Integer Format. If Hex Format is used the Immediate Data is treated
unsigned else it is treated as a signed integer.

\begin{verbatim}
call -12
call 0x12
\end{verbatim}

\elparagraph{12 bit}

12 bit Immediate Data must be specified as 16 bit Immediate Data which is later truncated by VASM to 12 bit Immediate Data. 12 bit Immediate Data is
always treated as unsigned.

\begin{verbatim}
add r0, r0, 5
\end{verbatim}

\subsubsection{Register Ranges}

VASM supports Register Ranges which are internally converted to 16 bit Integers. The syntax for Register Ranges is \verb|reg32:reg32|. It is allowed
to use Register Ranges everywhere where a 16 bit Integer is accepted. 

\begin{verbatim}
storema sp, r0:r5
loadma sp, r0:r5
\end{verbatim}

\subsection{Condition Codes}

VASM uses the official Condition Code Mnemonic as used in the official VA documentation. 

\begin{verbatim}
ldcallizic r0, _branch
\end{verbatim}

\subsection{Directives}

Directives always start with a dot \verb|.|. 

\section{Directives}

\subsection{.align}

Force the assembler to align the next byte. The syntax is \verb|.align int16 int16|. The first parameter specifies the alignment and the second parameter
specifies the filling byte. The assembler will emit filling bytes until the current address is divisible by the alignment. 

\begin{verbatim}
  ldjmp r0, _start  ; Address = 0
  .byte 0xFF        ; Address = 4
.label _start       
  stop              ; Address = 5
\end{verbatim}

This program would cause an alignment fault on the VA-CPU due to the STOP instruction not being correctly aligned. 

\begin{verbatim}
  ldjmp r0, _start ; Address = 0
  .byte 0xFF       ; Address = 4
.align 4 0x20      ; force alignment
.label _start
  stop             ; Address = 8
\end{verbatim}

Here the \verb|.align| forces the assembler to place the STOP instruction at a correctly aligned address. In the above example it will
put \verb|0x20| at address 5, 6 and 7 to be able to put the STOP instruction at address 8.

\subsection{.at}

Forces the assembler to skip to a given absolute address (inside the image file). The syntax is \verb|.at int32|. The given integer is treated as unsigned. 

\begin{verbatim}
.at 16
.dword @1  ; this dword is placed at address 16 
           ; (relative to image start address)
\end{verbatim}

\subsection{.byte}

Emit a raw byte. The syntax is \verb|.byte int16|. The given integer is treated as unsigned and truncated to 8 bits. 

\begin{verbatim}
.byte 0x46
.byte 0x00
.byte 0x05
.byte 0x00
\end{verbatim}

\subsection{.dword}

Emit a raw dword. The syntax is \verb|.dword int32|. 

\begin{verbatim}
.dword @157
\end{verbatim}

The above example actually describes a \verb|LOADI R0L, 0x05|.

\subsection{.label}

Places a label. The syntax is \verb|.label identifier|. A label assigns a name to a location in code. The label is later resolved by the assembler to its address. Labels
may also be placed before data. Lables must begin with an underscore.

\begin{verbatim}
ldjmp r0, _start
.label _data
.dword @198771
.label _start
  loadadr r0, _data
  loadid r0, r0, 1
\end{verbatim}

\subsection{.offset}

The \verb|.offset| directive forces the assembler to use the specified offset when doing label resolution. The default offset is 4096 as the Instruction Pointer
of the VA-CPU is initialized to 4096. If your code is loaded to a different address then you must change the offset accordingly. 

\begin{verbatim}
.offset 8192
stop	; at image address 0
        ; which is assumed to be address
        ; 0 + 8192 in memory
\end{verbatim}

\subsection{.word}

Emit a raw word. The syntax is \verb|.word int16|. 

\begin{verbatim}
.word 0x12AB
\end{verbatim}

\section{Pseudo Instructions}

VASM supports Pseudo Instructions to ease programming. 

\subsection{LDCALL}

The LDCALL Pseudo Instruction is a combination of the LOADADR and the CALL instruction. It loads the address of a label into a register and
performs a call to that address. 

\begin{verbatim}
ldcall r0, _go
\end{verbatim}

\subsection{LDJMP}

The LDJMP Pseudo Instruction is a combination of the LOADADR and the JMP instruction. It loads the address of a label into a register and
performs a jump to that address. 

\begin{verbatim}
ldjmp r0, _go
\end{verbatim}

\subsection{LOAD32}

Due to the Instruction Format of VA no instruction accepts more than 16 bit Immediate Data. Therefore, a 32 bit load must be assembled as two 16 bit loads.
This is done automatically by VASM through the LOAD32 Pseudo Instruction. Depending on the target register this Pseudo Instruction is assembled as either
two or four instructions. If the target register is not word-accessible VASM will use two additional XCHG instructions. 

\begin{verbatim}
load32 r8, @77000
\end{verbatim}

\subsection{LOADADR}

The LOADADR Pseudo Instruction allows loading an address of a label into a register. The syntax is \verb|loadadr reg32, ident|. This is instruction is always
assembled as two LOADI instructions and therefore the target register must be word-accessible. 

\begin{verbatim}
loadadr r0, _isPrime
...
.label _isPrime
...
\end{verbatim}

\subsection{POP}

The POP Pseudo Instruction allows popping a register from the stack where the SP register is implicitly used as the stack pointer. This instruction is assembled
as a LOADMA instruction. 

\begin{verbatim}
pop r0
\end{verbatim}

\subsection{PUSH}

The PUSH Pseudo Instruction allows pushing a register to the stack where the SP register is implicitly used as the stack pointer. This instruction is assembled
as a STOREMA instruction.

\begin{verbatim}
push r0
\end{verbatim}
