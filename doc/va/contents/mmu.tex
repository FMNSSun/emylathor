\section{MMU}

The MMU (Memory Management Unit) is responsible for managing virtual memory. Virtual memory allows use of virtual address that are translated
by the MMU to physical addresses. Using the MMU also enables protecting memory region from unprivileged and/or unwanted access. The MMU must be enabled
through the MMU Enabled Flag in the MR2 register. 

\subsection{Page directory}

The page directory is an array of 1024 addresses which point to page tables. The page directory is $1024 * 4 = 4KB$ in size. An address
of a page table may not be zero. If the address of an entry in the page directory is zero the MMU assumes that the corresponding page table does
not exist and will fire a Page Fault Exception. 

\subsection{Page table}

The page table is an array of 1024 page table entries. A page table entry contains 20 frame bits and 12 control bits.

\subsubsection{Page Table Entry}

\begin{tabular}{ | r | r | r | r | r | r | r | r | r | r | }
	\hline
	Frame Bits [20] & Reserved [4] & RU & EU & U & WP & WU & EP & PROT & PRESENT \\
	\hline
\end{tabular} \\

\stitle{Control bits}

\begin{itemize}
  \item RU - If the \emph{Read Unprivileged Only} Flag is set the page can only be read in unprivileged mode. 
  \item EU - If the \emph{Executable From Unprivileged} Flag is set the CPU is allowed to execute code contained in the page only if it is in privileged mode.
  \item U - The \emph{Used} Flag is set by the MMU when a page has been accessed.
  \item WP - If the \emph{Write From Privileged} Flag is set the page can be written to from privileged mode (otherwise only read access is allowed).
  \item WU - If the \emph{Write From Unprivileged} Flag is set the page can be written from unprivileged mode (otherwise only read access is allowed.
  \item EP - If the \emph{Executable From Privileged} Flag is set the CPU is allowed to execute code contained in the page only if it is in privileged mode.
  \item PROT - If the \emph{Protected} Flag is set the page can only be accessed (read or write) from privileged mode.
  \item PRESENT - If the \emph{Present} Flag is set the MMU assumes that the page is in memory. 
\end{itemize}

\subsection{Paging}

If the MMU is enabled all addresses are treated as virtual addresses. Before any access to memory these virtual addresses are translated by the MMU
to physical addresses (the MMU requires physical addresses). A physical page has a fixed size of 4KB. 

\stitle{Virtual address format}

\begin{tabular}{ | r | r | r | }
	\hline
	PDINDEX [10] & PTINDEX [10] & OFFSET [12] \\
	\hline
\end{tabular} \\

After the MMU has translated the virtual address it will check the control bits from the page table entry. 

\subsubsection{Page fault}

If the MMU detects an error it will fire a Page Fault Exception (more information about exceptions see~\elref{sec:interrupts}). The exact cause
of the Page Fault Exception is specified by an error code in the MR4 register and the MR3 register contains the virtual address that caused the Page Fault Exception. 

\stitle{Page Fault Error Codes}

\begin{itemize}
  \item $00_{16}$ - Success: No error.
  \item $01_{16}$ - Page Not Present: The present flag of the corresponding page was not set.
  \item $02_{16}$ - Access Violation: The MMU detected an access violation. (E.g. accessing a protected page from unprivileged mode)
  \item $03_{16}$ - Not Executable: The CPU tried to execute code in a non-executable area.
  \item $04_{16}$ - Entry Missing: No pointer to a page table was set in the page directory.
\end{itemize}


\subsubsection{Address translation}

To translate a virtual address the MMU loads the physical address of the page directory from the MR1 register. The MMU will
load the $PDINDEX$-th entry from the page directory to retrieve the physical address of the corresponding page table. Then the MMU will
load the $PTINDEX$-th entry from the page table. It will then replace the first 20 bits (from left to right) of the virtual address with the
Frame Bits from the page table entry.  
