\section{DMA}

DMA (Direct Memory Access) is a mechanism to allow hardware to directly access the physical memory without involving the CPU. 
As long as the CPU can access data solely through the cache DMA has no performance impact. However, the DMA controller and the cache controller
race for access to the physical memory. If a cache load or cache writeback has to be carried out while a DMA transfer is taking place the
cache controller has to wait for the completion of that transfer and thus may stall the CPU as the CPU needs data that currently can not be accessed.
The order in which DMA transfers are carried out is not defined. 

\subsection{DMA transfers}

DMA transfers include read (read from memory) or write (write to memory) transfers. Such transfers can only be requested by hardware. The CPU itself has no control
over the DMA controller. Before and after a transfer the DMA Controller forces the Cache Controller to do the necessary Cache Writebacks or Cache Loads
to guarantee memory consistency. 
